\section{Project Management (8 pages)}

This PPRP proposal covers the DUNE aspects of the overall LBNF/DUNE project, which is illustrated in figure \ref{fig:organogram}. This section describes how the DUNE aspects are managed and how they are integrated into the overall LBNF/DUNE project.

\subsection{Project Governance and Oversight} 

A UK Project Board, independent of the project has been set up to oversee the delivery of the project on behalf of STFC and BEIS and will be responsible for all governance, legal and funding matters, including use of contingency. The Project Board will be chaired by the Senior Responsible Owner (SRO) and incorporate representatives from STFC and BEIS.

The SRO will be supported by the STFC Programme Lead in managing the interface with DOE/Fermilab and, in conjunction with the UK PI, working to maintain project definition and ensure delivery of the UK scope. The UK LBNF/DUNE project management team, led by the UK PI and UK project manager, will attend to report on the project status, along with other key stakeholders. The Project Board will be supported by PD.

The UK Project Board will be supported by project oversight committees (OsC) to provide independent expert scrutiny of the UK projects on behalf of the SRO, covering all managerial, scientific and technical matters, and, for example, recommending access to contingency. Three project oversight committees will be established: one for DUNE, one for LBNF and one for PIP-II. The OsC for DUNE-UK will be a continuation of the existing DUNE OsC for the Pre-Construction phase project now underway.

STFC’s Impact Evaluation team in SPC will be an integral part of the project oversight arrangements and monitoring and evaluation plans will be co-developed with the LBNF/DUNE project management team to identify and capture the expected benefits in line with the new BEIS appraisal and evaluation framework. An outline plan will be developed towards the end of 2017 and reviewed in 2019 when the technical design and partner contributions to the project are finalised.

Project risk will be assessed, reviewed and actively managed throughout the project life cycle. The assessment will take account of technical, safety, financial and other risks, with preventative actions and contingency plans developed for the more significant risks. Risk management at the project level will be undertaken as an integral part of developing the full business case and the project management plan with the PMP for each sub-project documenting how risk is going to be managed. Requests for use of contingency will be brought to the Project Board for approval. Issues or risks will be monitored through the oversight committees and project board and may be escalated where considered to be above an “acceptable” level or corporate in nature. Examples include (a) where

Scientific and technical leadership of the LBNF/DUNE-UK project (including WP4/5) will be the responsibility of the UK PI. The project will be managed by the LBNF/DUNE-UK PM. The WP managers will report to the UK PM, who has responsibility for the day-to-day planning and execution of the project. The UK PM is responsible to the UK PI and accountable to the SRO for efficient delivery of the project to specification, cost and schedule.

The top level management body of the project is the project management board (PMC). It consists of the overall project PI, the overall project manager, the deputy PI, the WP managers (WP1-5), and their WP engineers (WP2/3) and WP project managers (WP4/5). It is chaired by the overall PI. The PMC reports to the Project Board and to the DUNE-UK Institutional Board (IB). The DUNE-UK IB includes a representative from each of the sixteen DUNE-UK institutions.


As usual, a project Oversight Committee (OsC) will be set up by STFC Programmes Directorate for DUNE to provide independent expert advice to the STFC SRO on progress and performance over the lifetime of the project. It will receive progress reports and presentations from the UK PI and Project Manager and monitor financial and operation performance. It will ensure that the project management plan enables progress and performance to be tracked effectively, review and comment on project progress in terms of specification, costs, schedule and risk and recommend corrective actions, advise on the use of Working Allowance and requests to access contingency, report to STFC on any material change in scope or specification proposed by the projects and any other issues that may materially affect the successful project delivery and undertake a final review of the projects on completion, including ‘lessons learned’.

The UK Principal Investigator (UK PI) provides the scientific leadership for the UK project and manages the technical interface with the international DUNE and LBNF projects (including PIP-II SRF), on all managerial scientific and technical issues. The UK PI is responsible for the timely delivery of the UK elements of the DUNE project within budget and to the required quality level.  The UK PI works closely with the STFC Programme Lead and reports formally to STFC via the Project Board and Oversight Committee (OSC). 

The UK Project Manager (PM) will have responsibility for the day-to-day planning and execution of the project according to this Project Management Plan. Work package managers will report to the UK PM, who will maintain the UK Resource Loaded Schedule for all UK project deliverables and report progress at each PMC meeting.  A hierarchical change-control procedure will be developed as part of the Project Management Plan.  Duties of the UK PM include: Conducting weekly WP managers meetings; maintaining and disseminating this Project Management Plan; monitoring of progress and resources; maintaining project schedule, finances and risk register; assisting with procurement activities; managing document production and authorisation; managing Quality Assurance guidelines; oversight and management of Health and Safety through the work packages.

The LBNF/DUNE-UK institutional board\cite{IB} is formed from one member from each collaborating institution and national laboratory department. It elects a chair and defines the governance of the LBNF/DUNE-UK collaboration. The Institutional Board will meet as needed. The DUNE-UK IB will: elect the IB chair; develop the DUNE-UK collaboration governance rules; elect a UK spokesperson to represent DUNE-UK on the DUNE Collaboration Resource Board; approve the Project  Management Plan; select the leadership of the DUNE-UK scientific and technical activities; approve the documentation ahead of submission to OsC/Gateway/PPRP project reviews; agrees the division of responsibilities and resources between the participating institutions and approves the WP management.

\subsection{Change Management}
The baseline change-control process for DUNE-UK involves a Change Control Board (CCB) which is the subsystem package managers plus the Project Manager (PM), Principal Investigator (PI) and Project Engineers. All significant technical, cost and schedule baseline changes will be reviewed by the board and a recommendation made to the PM for consideration. A log of all changes will be maintained by the Project Manager. In this instance ‘significant’ is defined as i) impacting more than one subsystem, ii) impacting L2 or L3 milestones by more than 1 month, or iii) impacting the cost of a subsystem by more than £50k.

\subsection{Risk Management}
for roy

\subsection{Health and Safety Management}

The safety of work conducted at collaborating institutions is the responsibility of each institution’s safety management, unless specified otherwise. For the DUNE-UK teams this responsibility belongs to each institute’s Principal Investigator, but the DUNE-UK Management coordinates proactively safety matters across the UK groups in order to ensure the safe delivery of the STFC-funded project. This means that the UK Project Manager and the UK Principal Investigator actively track how the institutes are controlling safety-related risks within their teams and monitor the training of their staff.

Besides the technical hazards encountered in the construction of any physics experiment, there are three additional challenges which must be noted: i) working under a foreign regulatory system; ii) working off-site, in a remote location; and iii) deploying a complex experiment. 
We address these in turn.

Some DUNE-UK work – and eventually most of the work – takes place in the US under a different regulatory framework. Technical work at the UK institutes requires that the teams be trained to the standard applicable there, under UK H\&S regulations.

The two major sites where DUNE-UK work is ongoing with UK participation (CERN and SURF) require significant travel from the UK. Each institution’s safety management are aware of the need for off-site working. All have implemented off-site working policies, and they will inform our off-site working procedures as appropriate, reviewing our general travel and accommodation plans. In parallel, the institute’s insurance teams will also be made aware of our work.

The Safety Plan will be developed and  applies the technical work conducted specifically at the institutes. The SHE arrangements, main hazard categories and controls at each institute are will be described in the PMP, which will be updated as risks and work develop.This plan will ensure that the local safety standards are appropriate for the work done in the UK, provides guidance and assistance when needed, and monitors the training needs for the UK staff. This information is kept up-to-date in a live document, which will be captured and made available to the OsC.

\subsection{Technical Reviews}
for roy

\subsection{Procurement Planning}
for roy

\subsection{Benefits Realisation and Impact Planning}
for roy??? use slides from BR WS

\subsection{QA/QC Planning}
for Roy, Justin, Giles, Simon, Andy

\subsection{Stakeholder Engagement and Communications}
for roy