\section{Project Management (8 pages)}

This PPRP proposal covers the DUNE aspects of the overall LBNF/DUNE project, which is illustrated in figure \ref{fig:organogram}. This section describes how the DUNE aspects are managed and how they are integrated into the overall LBNF/DUNE project.

\subsection{Project Governance and Oversight (1-2 pages)} 

A UK Project Board, independent of the project has been set up to oversee the delivery of the project on behalf of STFC and BEIS and will be responsible for all governance, legal and funding matters, including use of contingency. The Project Board will be chaired by the Senior Responsible Owner (SRO) and incorporate representatives from STFC and BEIS.

The SRO will be supported by the STFC Programme Lead in managing the interface with DOE/Fermilab and, in conjunction with the UK PI, working to maintain project definition and ensure delivery of the UK scope. The UK LBNF/DUNE project management team, led by the UK PI and UK project manager, will attend to report on the project status, along with other key stakeholders. The Project Board will be supported by PD.

The UK Project Board will be supported by project oversight committees (OsC) to provide independent expert scrutiny of the UK projects on behalf of the SRO, covering all managerial, scientific and technical matters, and, for example, recommending access to contingency. Three project oversight committees will be established: one for DUNE, one for LBNF and one for PIP-II. The OsC for DUNE-UK will be a continuation of the existing DUNE OsC for the Pre-Construction phase project now underway.

STFC’s Impact Evaluation team in SPC will be an integral part of the project oversight arrangements and monitoring and evaluation plans will be co-developed with the LBNF/DUNE project management team to identify and capture the expected benefits in line with the new BEIS appraisal and evaluation framework. An outline plan will be developed towards the end of 2017 and reviewed in 2019 when the technical design and partner contributions to the project are finalised.

Project risk will be assessed, reviewed and actively managed throughout the project life cycle. The assessment will take account of technical, safety, financial and other risks, with preventative actions and contingency plans developed for the more significant risks. Risk management at the project level will be undertaken as an integral part of developing the full business case and the project management plan with the PMP for each sub-project documenting how risk is going to be managed. Requests for use of contingency will be brought to the Project Board for approval. Issues or risks will be monitored through the oversight committees and project board and may be escalated where considered to be above an “acceptable” level or corporate in nature. Examples include (a) where

Scientific and technical leadership of the LBNF/DUNE-UK project (including WP4/5) will be the responsibility of the UK PI. The project will be managed by the LBNF/DUNE-UK PM. The WP managers will report to the UK PM, who has responsibility for the day-to-day planning and execution of the project. The UK PM is responsible to the UK PI and accountable to the SRO for efficient delivery of the project to specification, cost and schedule.

The top level management body of the project management board (PMC). It consists of the overall project PI, the overall project manager, the deputy PI, the WP managers (WP1-5), and their WP engineers (WP2/3) and WP project managers (WP4/5).
The top-level management body is the Project Management Committee (PMC), chaired by the UKPI. The PMC consists of the chair, a deputy chair, the UK PM and the sub-project leads of the five main sub-projects (Work Packages). The PMC reports to the Project Board and to the DUNE-UK Institutional Board (IB). The DUNE-UK IB includes a representative from each of the sixteen DUNE-UK institutions.

The roles of the different management bodies are outlined in Figure 2 and described in detail in section 4.
explain the roles, if IB\cite{IB}, PM, PI, PMC, OsC and Project board

\subsection{Change Management}
The baseline change-control process for DUNE-UK involves a Change Control Board (CCB) which is the subsystem package managers plus the Project Manager (PM), Principal Investigator (PI) and Project Engineers. All significant technical, cost and schedule baseline changes will be reviewed by the board and a recommendation made to the PM for consideration. A log of all changes will be maintained by the Project Manager. In this instance ‘significant’ is defined as i) impacting more than one subsystem, ii) impacting L2 or L3 milestones by more than 1 month, or iii) impacting the cost of a subsystem by more than £50k.

\subsection{Risk Management}
for roy

\subsection{Technical Reviews}
for roy

\subsection{Procurement Planning}
for roy

\subsection{Benefits Realisation and Impact Planning}
for roy???

\subsection{QA/QC Planning}
for Roy, Justin

\subsection{Stakeholder Engagement and Communications}
for roy