\subsection{WP1 Physics (5 pages)}

\subsubsection{Introduction}


{\bf Introduction and paragraphs about physics goes here}

The ultimate physics reach of DUNE depends critically on the performance of its reconstruction
software. Neutrino event reconstruction is the crucial step in which the precise imaging
capabilities of LAr-TPC detectors are exploited in order to deliver high-quality inputs to
the physics analyses. The automated reconstruction of neutrino interactions in LAr-TPC detectors
presents a significant challenge due to the complexity of the LAr-TPC images, which contain
a mixture of overlapping track-like and shower-like topologies. This remains a non-trivial and
unsolved problem, although significant advances have been made over the past couple of years.

The UK has a world-leading role in the development and delivery of advanced reconstruction
software for the LAr neutrino programme. The UK suite of reconstruction software spans the
low-level pattern recognition algorithms used to extract detailed information from LAr-TPC images,
as well as high-level particle identification and energy estimation techniques used to characterise
the reconstructed particle tracks and electromagnetic showers. In DUNE, the UK-led reconstruction tools
are key elements in the Far Detector physics and design studies, and will also provide the basis of
the ProtoDUNE data analysis. Figure X shows ... {\bf INSERT SOME FIGURES HERE !!!}

The cornerstone of the UK reconstruction software is the Pandora pattern recognition framework {\bf REFERENCE},
based on advanced particle flow techniques. Pattern recognition is the most critical
and challenging stage of the LAr-TPC reconstruction, and the UK's leadership in this area provides
a strong platform for the overall reconstruction and physics effort. Pandora promotes a proven
multi-algorithm approach to pattern recognition in which a large number of individual algorithms
gradually buiild up a picture of the event and, collectively, provide a robust reconstruction.
Pandora is the {\it de facto} reconstruction for multiple LAr-TPC experiments, including use for
the first physics results from MicroBooNE {\bf REFERENCE}

This work package will exploit the UK leadership in reconstruction software. The goal will be
develop and to deliver the main pattern recognition and high-level reconstruction software
to support the construction of the Far Detector, its commissioning and first data-taking,
and first physics results.

{\bf Need to write some paragraphs about computing and FD commissioning!}


.

\subsubsection{Work plan}

{\bf Insert a paragraph about physics here! Then discuss reconstruction as follows...}

The reconstruction sub-package will retain and exploit the UK's leading role in reconstruction software,
placing the UK at the heart of the DUNE physics programme. The effort will continue to focus on the
development and optimisation of pattern recognition and high-level reconstruction algorithms.
The goal of the next phase of development will be the delivery of the first production-ready software
to support the commissioning and first data-taking of the DUNE Far Detector and the first physics results.

The Pandora multi-algorithm approach to pattern recognition will remain the backbone of the
UK reconstruction effort. The next phase of development will add new levels of sophistication,
including the application of calorimetric information, new approaches to 3D reconstruction,
and the use of iterative techniques, which are uniquely available within the Pandora framework.
A key aspect of the new programme of work will be the harnessing of deep-learning architectures,
which have the potential to create a step change in LAr-TPC reconstruction software.

Much of the current effort has focused on accelerator neutrino reconstruction software to support
oscillation physics studies. In the next phase of development, this scope will be expanded to
include atmospheric neutrinos and cosmic-ray data, which  are expected to provide the first
Far Detector data samples. In addition, the UK reconstruction tools will be expanded to encompass
dual-phase as well as single-phase technology.

{ \bf Need to discuss ProtoDUNE !!!}

\subsubsection{WP1.1 Oscillation Physics}

\subsubsection{WP1.2 Reconstruction Software}

\begin{itemize}

\item {Core pattern recognition software} (Warwick and Lancaster)

\item {Incorporation of Deep Learning within pattern recognition} (Warwick)

The harnessing of deep-learning approaches to characterise event topologies and steer pattern recognition
algorithms will be led by the Warwick group. Algorithms will typically persist descriptions of the relevant
image/problem for examination offline, using Python to control packages such as TensorFlow, PyTorch or similar.
The first use of deep learning will be an analysis of the hits that provide the input to the pattern recognition.
Information will be attached to each hit to indicate their probability of originating from a track-like or
shower-like particle. Once this information is available in Pandora, it will form the basis of new algorithms
for reconstructing track-like and shower-like topologies. Deep learning approaches will also be exploited
to improve the identification of neutrino interaction vertices and to reconstruct the positions of downstream,
secondary vertices. Knowledge of secondary ineraction vertices will be valuable for guiding the pattern
recognition of the most complex events at DUNE. A novel, iterative chain of algorithms will be developed to
analyse any given vertex region and efficiently reconstruct the particles emerging from the vertex.

\item {High-level reconstruction tools} (Cambridge)

\item {Non-accelerator and dual-phase reconstruction} (Lancaster and Warwick)

\item {Core technical work}

\end{itemize}



\subsubsection{WP1.3 Computing}

\subsubsection{WP1.4 Far Detector Commissioning}

\subsubsection{Deliverables and Milestones}

\subsubsection{Business Case}

What are the benefits for the UK --- what do I write here?
