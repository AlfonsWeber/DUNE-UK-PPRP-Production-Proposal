\section{Objectives and Strategy}

The BEIS investment of £65M is to enable the UK to contribute to four key areas of the LBNF and DUNE international construction programme:
\begin{itemize}
    \item PIP-II SRF: The UK will construct and deliver the four high-beta superconducting SRF cavities and assembled cryomodules that make up the final stage of the new PIP-II Linac at Fermilab
    \item LBNF Target: The UK will design, construct and deliver the target for the neutrino beam and elements of the supporting infrastructure, including the complex remote handling system.
    \item DUNE Far Detector:
    \begin{itemize}
        \item DUNE APAs: The UK will construct and deliver 150 of the 300 large (6.0 m x 2.3 m) anode plane assemblies (APAs) for two DUNE Far Detector modules;
        \item DUNE DAQ: the UK will design, construct and deliver the back-end high-speed data acquisition (DAQ) system for two DUNE Far Detector modules.
    \end{itemize}
\end{itemize}
Additionally, the UK is leading the way in the automatic reconstruction of the complex data delivered by the detectors. This is an essential tool, which needs further development for the physics exploitation of the experiment.

This investment allows the UK to take a major stake in the DUNE Far Detector construction, securing future access to the best physics for the UK, and to play a leading role in the LBNF beam line and associated PIP-II accelerator development.

The bulk of the UK’s deliverables will be designed and manufactured in the UK, providing opportunities for UK industry to build capability in new and developing technologies, for example, precision engineering, DAQ/electronics, cryogenics and accelerator-based applications. It also offers the possibility to build strong partnerships between UK and developing nations, particularly in Latin America\footnote{not part of this proposal}.

The benefits to the UK can be summarised under the following headlines:
\begin{itemize}
    \item Forge a strategic partnership with the US on a high profile international project.
    \item Build a strong partnership between UK and developing nations especially in Latin America.
    \item Industrial engagement with UK industry and capability building.
    \item Skills retention and development of STEM workforce
\end{itemize}

The DUNE UK Construction Project has been structured around the above three main deliverables: preparation for the science, high speed DAQ and APA construction. However, the DUNE collaboration has not made a final decision about the technology to be deployed for different far detector modules. As mentioned earlier, it is generally accepted that the first module will be using the SP technology, while the second and third module could be either SP or DP. Our strategy to deal with this uncertainty is as follows:
\begin{itemize}
    \item {\bf We plan to build one half of the APAs for two single phase detectors (150).}\\ 
    The APA production line would be stopped after 75 APAs, if only one SP detector would be built. The unused capital and resources for the production of the second batch of 75 APAs,
    \todo{Justin, how much would this save?}
    would become available to broaden the scope of the project and/or to cover a shortfall of working margin. A new proposal would be submitted to STFC to use the saving in such a way as to maximise the return for the UK. Below we list project items that would come under consideration in this scenario and which have been taken out of the scope of the current proposal due to overall capital/resource limitations:
    \begin{itemize}
        \item The UK has developed reflector foil technology, which could be integrated into the cathode plane assembly (CPA). It would increase the overall light level and homogeneity of the system and substantially improve the detector capability for low energy signatures from, for example, supernova neutrinos.
        \item The UK has taken a lead in the design of the near detector and is developing the technology for a high pressure gas TPC (HPgTPC) which is very likely to form part of the DUNE near detector complex. We could contribute to the construction of a HPgTPC for the near detector.
        \item The exact scope of the WP4 (LBNF target, which is not part of this proposal) has not been defined and is subject to the phase 1 design. Additional resources available from DUNE, could be made available to WP4.
        \item ...\todo{Anything else?}
    \end{itemize}
    \item {\bf We plan to deliver the DAQ for one SP and one DP detector.}\\ 
    The back-end DAQ for both technologies is the same. However, the DP electronics contains functionality that has to be covered by the DAQ in the SP configuration. As such the DAQ for the SP detector is roughly £2M \todo{Giles, is that the right number} more expensive. If the two first detectors are SP, the collaboration as a whole will have to provide the missing resources as the capital/resource limitations of the project do not allow the UK to provide the DAQ for two SP detectors.
\end{itemize}

