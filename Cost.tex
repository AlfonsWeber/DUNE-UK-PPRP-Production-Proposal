\section{Resource Request and Cost}

The project cost tables can be found in Appendix \ref{app:A}. Here we summarise the assumption used in deriving the costings as well as options for changing the scope.

\subsection{Assumptions in Determining Working Margin}

The working margin (WM) for the project have been determined on a risk based analysis. We have also profiled the WM in time according to when it is most likely needed.

A separate WM has been calculated for the equipment cost. Depending on the confidence of the cost estimates a WM of in between 5 and 25 \% has been used.
{\color{red} is this true??? Giles/Simon and Alan/Justin: give us numbers!}
\begin{itemize}
    \item  5\% on equipment for which we have several quotes
    \item 10\% on equipment for which we have previous experience
    \item 25\% on equipment for which we have an expert guess
\end{itemize}

\subsection{Costing Assumptions and Methods}

We have used the following assumptions when costing the proposal
\begin{itemize}
    \item This is a capital project, so we assumed that no overhead have to be paid on STFC staff working on WP2/3. 
    \item It is still unclear, if we can recover VAT on equipment to be delivered to the US. We have therefore costed 20\% VAT into all equipment purchase and have included the possible recovery of VAT as a "negative risk" in the risk register.
    \item We make no difference between the traditionally used working allowance (held by the project) and contingency (held by the office) in this project. Instead we are using an overall working margin, which is included in the overall project cost.\footnote{A draft document explaining the use and management of the working margin is available on the TWiki.}
    \item Costs have not been indexed for inflation. However, wage increased due to pay progression have been included.
    \item STFC is only partially funding academics on the consolidated grants. We have therefore used a scaling factor of funded fraction divided by the research time (assumed to be 15\%/60\% for all academics) for their off-project cost, but given their actual FTE involvement in the project.
    \item We have assumed that all STFC staff requested by the collaborating institutions have been awarded, when producing the tables. However, we assigned a substantial WM to account for the fact that these post may not be awarded. (10\% of the fEC, if the person is currently funded on the CG and 80\%, if the person is requested on the CG, but not currently funded on the CG). We will provided updated tables once the outcome of the current PPGP round is known.
    \item ...{\color{red} What else?}
\end{itemize}

\subsection{Assumptions Relating to Capital}

We have assumed that all cost in WP2 and WP3 can be capitalised except for the WP manager. The WP engineer can be capitalise. We currently assume that the effort going in producing the software asset (PANDORA WP1.1, \todo{WP to be checked}) is resource, but this could be classified as capital as well. We assume the per head part of the CF contribution to be resource, while the equipment levy is capital.


\subsection{Common Fund}
\label{sec:CF}

As a recent development, the DUNE collaboration/project requires a construction common fund (CF) to cover the parts of the installation of equipment as well as common infrastructure, which would be difficult for a collaborator to receive funding for. There has been now concrete proposal put forward or agreed at the time of writing this proposal, but the model discussed between the US DUNE/LBNF project and the collaboration is likely to evolve the following assumptions:
\begin{itemize}
    \item Common equipment and infrastructure, which is not part of the facility and which is not easily funded by collaborators is part of the common fund. This could for example include racks, connection between racks and the facility infrastructure, control room, support structure for the APAs, ...
    \item The technical coordination team, in charge of the installation planning, and general purpose technicians and riggers that are needed during the installation are covered by the CF.
    \item The costing model put forward to cover this is suggested to have two components.
    \begin{itemize}
        \item k\$ 7-10 per PhD physicist (or similar) per year over a 5 year period
        \item 10-15\% of the value of core capital contributing to the equipment delivered to DUNE is charged as an additional CF contribution.
    \end{itemize}
    \item it is assumed that CF contributions can be provided in cash or in kind.
\end{itemize}
We have costed our CF contributions at an average level of the above numbers: 8.5k\$ per PhD and year (assuming 100 UK PhD collaborators) and 13\% of the equipment cost. We have informal agreement with the DUNE/LBNF project that this will not change when the real costs have been calculated and agreed by the international Resource Review Board. We also have an agreed set of in-kind contributions, which are listed in WP2 and WP3. The CF contributions are summarised in table \ref{tab:CF}.\footnote{The exchange rate of 1.2 US\$/1£ has been used.} 
\todo{Table needs to be agreed and checked!}

\begin{table}
{
    \centering
    \begin{tabular}{|c||c|c|c|c|c|c|c|c||r|}
        \cline{2-10} \multicolumn{1}{c|}{\ } & \multicolumn{9}{c|}{cost in k£}\\
         \hline
           & 19/20 & 20/21 & 21/22 & 22/23 & 23/24 & 24/25 & 25/26 & 26/27 & total\\
         \hline\hline
         WP2     & & 112 & 112 & 112 & 112 & 112 & & & 560  \\ % 13 of core equipment cost
         WP3     & & 167 & 167 & 167 & 167 & 167 & & & 835  \\ % 13 of core equipment cost
         Per PhD & & 708 & 708 & 708 & 708 & 708 & & & 3,540 \\
         \hline
         total  &  & 987 & 987 & 987 & 987 & 987 & & & 4,935 \\
         \hline\hline
         cash          &     & 492 & 507 & 357 & 642 & 717 &     &     & 2,715 \\ % adjusted to make up total
         WP2 equipment &     & 135 & 135 &     &     &     &     &     & 270 \\ % racks for 2 caverns
         WP1 OC        &     &     &     & 150 & 150 & 150 &     &     & 450 \\ % let's assume 3 years of somebody. too late?
         WP2 TC        & 150 & 225 & 225 & 225 &  75 &     &     &     & 900 \\ %6 years of Tim & TD Eng at 150k£/FTE*y
         WP3 TC        &     & 120 & 120 & 120 & 120 & 120 &     &     & 600 \\ % 4 years of Alan Muir at 150k
         \hline         
         total         & 150 & 835 & 987 & 987 & 987 & 987 &     &     & 4,935 \\
         \hline
    \end{tabular}
    \caption{Breakdown of common fund contributions. The top half of the table shows how much we owe the experiment. The bottom half shows, how we want to cover this with cash and in-kind contributions. It is worth noting that the effort cost in the table are the value of our contributions to the CF, which is higher than the cost to the UK project. For example, the value of an engineering staff year is k\$ 200.}
    \label{tab:CF}
}
\end{table}

\subsection{De-scope Options}
We believe that we have costed the project conservatively and that it can fit into the funding envelope outlined in the business case to BEIS. This is the minimum level of scope that is outlined in the business case and can deliver the benefits to the UK. If the committee does not feel that we can deliver the scope of the proposal within the funding envelope, we could consider the following de-scoping options.

\subsubsection{Timing System}
The timing system (WP2.X) is a standalone system that is however integral to the DAQ. It could potentially be delivered by a non-UK collaborator. There is so far no institute or country that has expressed interest in doing so. Not including the timing system in the scope of the project would safe {\color{red} so many £}.
However, there are substantial problems in doing so:
\begin{itemize}
    \item There is no other country or institute interesting in providing the timing system
    \item The UK has already successfully developed and delivered the timing system for protoDUNE. Any new institutions, picking this up would probably want to change the design to match their expertise, which would increase the risk to the overall project.
    \item The international DUNE collaboration is expecting the UK collaborators to deliver the DAQ including the timing system. Withdrawing from this informal agreement would have substantial reputation risks, especially as there is nobody else interested in providing this sub-system.
    \item The timing system would be manufactured by UK industry. Removing it from the scope of the system would reduce the return for the UK.
\end{itemize}

\subsubsection{Fewer APAs}
An easy way to reduce the overall cost is to reduce the overall number of APAs that the UK is producing assuming that another body (instite/country) would increase their APA production correspondingly. For each 10 APAs the total project cost would go down by {\color{red} M£ 1???, Justin to check number}. 
However, there are substantial problems in doing so:
\begin{itemize}
    \item The only other country interested in producing APAs is the US. There is not enough money in the US project to cover this.
    \item The international DUNE collaboration is expecting the UK collaborators to deliver half of the APAs for two single phase far detector modules. This is also consistent with the business case put forward to BEIS. Withdrawing from this informal agreement would have substantial reputation risks.
    \item \color{red} More arguments from Stefan.
\end{itemize}

\subsubsection{Ignoring the computing requirements}
We could reduce the grid related activities of the project by 10 SY saving £ 1M.
However, there are substantial problems in doing so:
\begin{itemize}
    \item Pete would be furious.
    \item We might have software and hardware, but no way to analyse the data as we can't efficiently run jobs
    \item UK is leading the development of computing model for DUNE
    \item FNAL/DUNE would be lost without us.
\end{itemize}

\todo{Andy and Pete to provide better arguments.}

\subsection{Re-Scoping Options}

There have been several pressured to the total cost of the project. The two main coming from the construction common fund contributions and the fact that we might have to pay VAT on materials to be exported to the US. It was not know at the time of formulating the business case to BEIS that these had to be paid, but they have been costed into the proposal. However, to do so, we had to remove some important components of the proposal.

At the time of submitting the proposal, there is still substantial uncertainty on whether VAT has to be paid or not. In the case, some of the risks don't materialise WM might become available to be used in the project. This WM might be used to cover risks or increase scope in other parts of the program (WP4/5) or change the scope of the work-packages in this proposal. A proposal, on how to use this flexibility would be put forward by the PMC for approval by the PB, if this happens after project approval.

Within DUNE (WP1-3) we could recover the following activities.

\subsubsection{TPC Photon Reflector Foils}
\todo{We need to spell this out a little. one paragraph}
 
This would require an additional PPRP proposal
 
\subsubsection{DAQ for 2 SP modules}
\todo{We need to spell this out a little. one paragraph}

This is just a shift of resources, and could be approved by the PB.
