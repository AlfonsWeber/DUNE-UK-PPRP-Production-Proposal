\subsection{Costing Assumptions and Methods}

We have used the following assumptions
\begin{itemize}
    \item This is a capital project, so we assumed that no overhead have to be paid on STFC staff working on WP2/3. 
    \item It is still unclear, if we can recover VAT on equipment to be delivered to the US. We have therefore costed 20\% VAT into all equipment purchase and have included the possible recovery of VAT as a "negative risk" in the risk register.
    \item We make no difference between the traditionally used working allowance (held by the project) and contingency (held by the office) in this project. Instead we are using an overall working margin, which is included in the overall project cost.\footnote{A draft document explaining the use and management of the working margin is available on request the TWiki.}
    \item Costs have not been indexed for inflation. However, wage increased due to pay progression have been included.
    \item STFC is only partially funding academics on the consolidated grants. We have therefore used a scaling factor of funded fraction divided by the research time (assumed to be 15\%/60\% for all academics) for their cost, but given their actual FTE involvement in the project.
    \item \color{red}What else?
\end{itemize}



\subsection{Assumptions in Determining Working Margin}

The working margin (WM) for the project have been determined on a strictly risk based analysis. We have also profiled the WM according to when it is most likely to be needed.

The equipment cost already includes a risk element calculated as follows {\color{red} is this true??? should we separate this out?}
\begin{itemize}
    \item 10\% on equipment for which we have a quote
    \item 20\% on equipment for which we have previous experience
    \item 50\% on equipment for which we have an expert guess
\end{itemize}

\subsection{Assumptions Relating to Capital}

We have assumed that all cost in WP2 and WP3 can be capitalised except for the WP manager. The WP engineer can be capitalise.
We currently assume that the effort going in producing the software asset (PANDORA WP1.1, {\color{red} WP to be checked}) is resource, but this could be classified as capital as well. We assume the per head part of the CF contribution to be resource, while the equipment levy is capital.


\subsection{Common Fund}

As a recent development, the DUNE collaboration/project requires a construction common fund (CF) to cover the parts of the installation of equipment as well as common infrastructure, which would be difficult for a collaborator to receive funding for. There has been now concrete proposal put forward or agreed at the time of writing this proposal, but the model discussed between the US DUNE/LBNF project and the collaboration is likely to evolve the following assumptions:
\begin{itemize}
    \item Common equipment and infrastructure, which is not part of the facility and which is not easily funded by collaborators is part of the common fund. This could for example include racks, connection between racks and the facility infrastructure, control room, support structure for the APAs, ...
    \item The technical coordination team, in charge of the installation planning, and general purpose technicians and riggers that are needed during the installation are covered by the CF.
    \item The costing model put forward to cover this is suggested to have two components.
    \begin{itemize}
        \item 7-10 k\$ per PhD physicist (or similar) per year over a 5 year period
        \item 10-15\% of the value of core capital contributing to the equipment delivered to DUNE is charged as an additional CF contribution.
    \end{itemize}
    \item it is assumed that CF contributions can be provided in cash or in kind.
\end{itemize}
We have costed our CF contributions at an average level of the above numbers: 8.5k\$ per PhD and year (assuming 100 UK PhD collaborators) and 13\% of the equipment cost. We have informal agreement with the DUNE/LBNF project that this will not change when the real costs have been calculated and agreed by the international Resource Review Board. We also have an agreed set of in-kind contributions, which are listed in WP2 and WP3.

