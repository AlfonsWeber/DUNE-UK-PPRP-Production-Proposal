LBNF/DUNE is an approved project with a defined US funding profile. The major milestones through the ten-year construction period are:
\begin{itemize}
     \item {\bf 2017:} start of underground construction at SURF, South Dakota;
     \item {\bf 2018:} operation of the large-scale protoDUNE engineering prototype at CERN; 
     \item {\bf 2019:} provisional international resource matrix for DUNE construction;
     \item {\bf 2019:} technical design report (TDR) for the DUNE detectors (to be reviewed in Q4 2019);
     \item {\bf 2020:} start of production of far detector components;
     \item {\bf 2021:} start of installation the first 10\,kt far detector module at SURF; 
     \item {\bf 2024:} start of physics operation of the first 10\,kt far detector module;
     \item {\bf 2026:} start of 1.2\,MW beam operation with the first two 10\,kt far detector modules.  
\end{itemize}
As of October 2018, the DUNE collaboration consists of around 1100 collaborators from 175 institutions from 31 nations. DUNE/LBNF received strong international support. 

\subsection{Funding Situation}
Only a few countries outside the US have made firm funding commitments towards DUNE/LBNF. But there is a very strong and wide international support and many positive signs from around the world: 
Brazil received R\&D funding from FAPESP; China is providing in-kind contribution to the LBNF beam line; France is funding protoDUNE and large parts for the dual phase detector; Germany received R\&D funding for the near detector and has included it in its future strategy; India has agreed more than \$ 100M of in-kind contribution to PIP-II; Italy has agreed PIP-II contributions and is discussing DUNE involvement; Spain provided R\&D funding for protoDUNE; and Switzerland made a large investment in protoDUNE and is expected to be a major contributor to the near and a Dual Phase far detector.

The funding matrix will be established over the next year in time for CD2. The UK commitment via BEIS was the first of a series of contributions to establish the full funding matrix.


\subsection{Decision Points}

There a some high level decisions, which will influence the UK project. There are two technologies currently considered to be used for the different DUNE FD modules: i) single phase (SP) TPC using the APA readout and ii) dual phase (DP) TPC, which uses Charge Readout Planes (CRPs) in the gas for additional signal amplification. While no formal decision has been made, it is more than likely that the first module will be SP. The formal decision will be made in spring 2019. The second and third modules could be either single or dual phase. The decision on these technologies should be made in time for the DUNE CD2 ({\color{red}give date}) and will depend on the results of running the protoDUNE detectors at CERN and the international funding matrix.

It is likely that at least two SP and one DP modules will be constructed.

